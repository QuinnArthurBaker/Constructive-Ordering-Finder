\documentclass{article}
\usepackage[utf8]{inputenc}
\usepackage{amsmath}
\usepackage{amsthm}
\usepackage{amssymb}
\usepackage{amsfonts}
\usepackage{tikz}
\def\checkmark{\tikz\fill[scale=0.4](0,.35) -- (.25,0) -- (1,.7) -- (.25,.15) -- cycle;} 
\newcommand{\ZZ}{\mathbb{Z}}
\newcommand{\ZnZ}{\ZZ/n\ZZ}
\newcommand{\FZ}{F_{\preceq}\left(\ZZ/n\ZZ\right)}
\newtheorem{conjecture}{Conjecture}
\newtheorem{theorem}{Theorem}
\newtheorem{lemma}{Lemma}
\newtheorem{definition}{Definition}
\title{Introduction to Factorial Subsets}
\author{Zack Baker }
\date{June 2017}

\begin{document}

\maketitle

\section{Introduction}

A \textit{group} $G$ is a set of elements together with a binary operation $\cdot$ such that the following properties hold:
\begin{enumerate}
\item \textbf{Closure} of the set under the binary operation: For any $a,b\in G$, $a\cdot b$ is also an element of $G$.
\item \textbf{Associativity:} for any $a,b,c\in G$, $(a\cdot b)\cdot c = a\cdot(b\cdot c)$.
\item There exists an \textbf{identity} element $e \in G$ such that for all $a\in G$, $a\cdot e = e\cdot a = a$.
\item Each element $a\in G$ has an \textbf{inverse} $b\in G$ such that $a\cdot b=b\cdot a = e$.
\end{enumerate}

When the binary operation is understood, and we are working in a multiplicative group, we will often use concatenation in place of writing the binary operation between two elements. For example, if we want to represent $a\cdot b$ for $a,b\in G$ we may simply write $ab$.

Group theory aims to understand the relations between elements of a group. This paper aims to further analyze the relationship between elements of a group by extending the concept of the factorial from the integers to any group.

For $n$ a non-negative integer, the \textit{factorial} of $n$, written $n!$ is defined as $n\times (n-1) \times \ldots \times 2 \times 1$, with the special case $0!=1$. Thus, $1! = 1$, $2!=2\times 1 = 2$, $3! = 3\times 2 \times 1 = 6$, and so on. We seek to extend the factorial by imposing an ordering on the underlying set of the group.




\section{Definitions}
Let $G$ be a finite group under the binary operator $\oplus$. To begin, we must define a binary relation $\preceq$ on $G$, where $\preceq$ is a total ordering. If the group's underlying set is numeric, and if $\preceq$ is the same as the standard $\leq$ ordering on the set, then we call $\preceq$ the \textit{natural ordering} for $G$. 


We define the factorial of $g_i$, $g_i! = \bigoplus_{k=0}^{i} g_k = g_0 \oplus g_1 \oplus g_2 \oplus \ldots \oplus g_k$, where $g_0\preceq g_1\preceq g_2 \preceq \ldots \preceq g_{k-1} \preceq g_k$ and there are no elements $g\in G\setminus\{g_0,...,,g_k\}$ such that $g_0\preceq g\preceq g_k$.
\begin{definition}
Given a group $G$ with some total ordering $\preceq$, we define the \textit{factorial subset} of $G$, denoted $F_\preceq(G)$, as $\{g_i! | g_i \in G\}$. In other words, the factorial subset of a group is the set of the factorial of each of that group's elements. If $F_\preceq(G)$ is a subset of $G$, then we say it is a \textit{factorial subset} of $G$. 
\end{definition}

\textbf{Example:} The Klein 4-group is a small, common group suitable for demonstrating the concepts established above. The Klein 4-group is defined as $\{e,a,b,c\}$, where $e$ is the identity, $a^2 = b^2 = c^2 = e$, $ab = c$, $ac = b$, and
$bc = a$. Define $\preceq$ such that $e\preceq a \preceq b \preceq c$. Then, $e! = e$, $a! = ea = a$, $b! = eab$ = $ab = c$, and $c! = eabc$ = $cc$ = $e$. Then, the factorial subset of the Klein 4-group, $F_\preceq(V_4)$, is $\{e!,a!,b!,c!\}$, or $\{e,a,c\}$. Since the factorial subset has 3 elements and $V_4$ has 4, clearly the subset is not a subgroup of $V_4$. In fact, no orderings of $V_4$ produce a subgroup
\begin{theorem}
For any ordering $\preceq$, $F_\preceq(V_4)$ is not a subgroup of $V_4$.
\end{theorem}
\begin{proof}
Let $x,y,z,w$ be distinct elements of $V_4$. Then, $F_\preceq(V_4) = \{x!,y!,z!,w!\}$. Let $w$ be the greatest element. Then, $w! = e\cdot a\cdot b\cdot c = e$. We will show that some element is always duplicated in $F_\preceq(V_4)$. We have 4 cases:
\begin{itemize}
    \item[Case 1:] Let $x=e$. Then, $x! = e$, and $e$ is duplicated
    \item[Case 2:] Let $y=e$. Then, $y! = x!\cdot y = x!\cdot e = x!$. Thus, $x!$ is duplicated
    \item[Case 3:] Let $z=e$. Then, $z! = y!\cdot z = y!\cdot e = y!$. Thus, $y!$ is duplicated.
    \item[Case 4:] Let $w=e$. Then, $w! = z! \cdot w = z!\cdot e = z!$. Thus, $z!$ is duplicated
\end{itemize}
Therefore, in all cases, $F_\preceq(V_4) = \{e,x,y\}$, where $x,y$ are distinct elements of $V_4$. 
\end{proof}

\section{Factorial Subsets of the Additive Group of Integers modulo $n$}
In this section, we will specifically examine the factorial subsets and subgroups of the additive group of integers modulo $n$, $\mathbb{Z}/n\mathbb{Z}$. 
\subsection{Factorial Subsets with the natural ordering of $\ZnZ$}
Let $\ZnZ$ be the additive group of integers modulo $n$, and let $\ZnZ$ be ordered by the natural ordering.  The following table describes $F_\preceq(\ZnZ)$ for small $n$. \\ \\

\begin{tabular}{|c|c|c|}
\hline
$n$ & $F_\preceq(\ZZ/n\ZZ)$ & \textbf{Is $F_\preceq(\ZZ/n\ZZ)$ a subgroup of $\ZZ/n\ZZ$?}\\
\hline
2 &$\{0,1\}$ & \checkmark \\
\hline
3 &\{0,1\} & \textbf{X}\\
\hline
4 &\{0,1,2,3\} & \checkmark \\
\hline
5 &\{0,1,3\} & \textbf{X} \\
\hline
6 & \{0,1,3,4\}& \textbf{X}\\
\hline
7 & \{0,1,3,6\}& \textbf{X} \\
\hline
8 &\{0,1,2,3,4,5,6,7\} & \checkmark \\
\hline
9 &\{0,1,3,6\} &\textbf{X} \\
\hline
10 &\{0,1,3,5,6,8\} & \textbf{X} \\

\hline

\end{tabular} \vspace{0.1in}
\\As we can see, $F_\preceq(\ZZ/n\ZZ)$ is only a subgroup of $\ZZ/n\ZZ$ if $n$ is 2, 4 or 8. Continuing these calculations up to $n=10000$, $F_\preceq(\ZZ/n\ZZ)$ is only a subgroup if $n = 2^k$, for some positive integer $k$, and for all $n=2^k$, $F_\preceq(\ZZ/n\ZZ) = \ZZ/n\ZZ$
\begin{conjecture}
For $\preceq$ the natural ordering, $\FZ$ is equivalent to $\ZZ/n\ZZ$ if and only if $n=2^k$ for some positive integer $k$. 
\end{conjecture}

\subsection{All Factorial Subgroups of $\ZZ/n\ZZ$}
In the previous section we examined a single ordering for many values of $n$. Conversely in this section, we will look at many orderings for few $n$. We ask for some $n$, how many orderings of $\ZZ/n\ZZ$ produce factorial subgroups? Since there are $n!$ ways of ordering $n$ elements, there will be many different orderings to check. The following table describes the number of factorial subgroups produced by $\ZZ/n\ZZ$

\begin{tabular}{|c|c|}
\hline
$n$ & Number of orderings for which $F_\preceq(\ZZ/n\ZZ)$ is a subgroup of $\ZZ/n\ZZ$\\
\hline
2 & 1\\
\hline
3 & 0\\
\hline
4 & 2\\
\hline
5 & 0\\
\hline
6 & 4\\
\hline
7 & 0\\
\hline
8 & 24\\
\hline
9 & 0\\
\hline
10 & 288\\
\hline
11 & 0\\
\hline
12 & 3856\\
\hline
\end{tabular} \vspace{0.1in}
\\Curiously, taking the non-zero entries as a sequence replicates the known integer sequence A14159.
\par Since for $\ZnZ$ there are $n!$ possible orderings, this quickly makes computation impossible for larger values of $n$. There are several observations which reduce the total number of orderings needed to check.
\begin{theorem}
$\FZ$ is a subgroup of $\ZnZ$ if and only if $\FZ$ = $\ZnZ$
\end{theorem}
\begin{proof}
For some $n$, let $a_i$ be the factors of $n$. Then, $<a_i>$ generates a subgroup of $\ZnZ$. All elements of all subgroups are of the form $n\cdot a_i$, where $a_i$ is the generating element. Assume for all elements $g_i\preceq 1$, $g_i = n\cdot a_i$, and let $g_k$ directly precede 1. Then, 1! = $g_k!\cdot a_i + 1$. If $a_i \neq 1$, then $g_k! \cdot a_i + 1$ is not a multiple of $a_i$, so $g_k!\cdot a_i + 1$ isn't in $<a_i>$. The only $a_i$ for which $g_k!\cdot a_i + 1$ is a multiple of is 1. Since the subgroup generated by 1, $<1>$, is the entire group, $\FZ$ must be equal to $\ZnZ$ to be a subgroup of $\ZnZ$. 
\end{proof}
\begin{lemma}
For any $n$, if $0$ is not the smallest element in $\ZnZ$, then $\FZ$ \neq $\ZnZ$. 
\end{lemma}
\begin{proof}
This will be a proof by contradiction. Assume there exists some element $g_i$ directly preceding 0. Then, $0! = g_i! + 0 = g_i!$. Since $g_i!$ is repeated, the size of $\FZ$ must be smaller than $\ZnZ$, so $\FZ \neq \ZnZ$. 
\end{proof}
\begin{lemma}
If $n$ is odd, $\FZ \neq \ZnZ$. 
\end{lemma}
\begin{proof}
Let $g_n$ be the largest element of $\ZnZ$. Then, $g_n! = \sum_{k=0}^{n-1} k = T_{n-1} \equiv T_n - n \equiv T_n \pmod{n}$, where $T_n$ is the $n$th triangular number. If $n$ is odd, then $T_n \pmod{n}$ is 0. The triangular numbers are defined as $T_n = \sum_{i=0}^{n} = 0 + 1 + 2 + \ldots + n-1 + n$. If $n$ is odd, the center of this sum looks like $\ldots + \frac{n-1}{2} + \frac{n+1}{2} + \ldots$. In $\ZnZ$, each element can be paired with its inverse, as there are an even number of terms in the sum, excluding 0. Thus, $T_n \pmod{n} \equiv 0$ for odd $n$. Then, 0 is duplicated in $\FZ$, so $\FZ \neq \ZnZ$.
\end{proof} 
\begin{lemma}
For even $n$, $n>2$, $\frac{n}{2}$ cannot be the greatest element.
\end{lemma}
\begin{proof}
This will be a proof by contradiction. Assume $\frac{n}{2}$ is the greatest element of $\ZnZ$. Then, since the factorial of the largest element of $\ZnZ$ is $T_n$, $\frac{n}{2}! = T_n$. For even $n$, $T_n = \frac{n}{2}$. The $n$th triangular number is defined as $0 + 1 + 2 + \ldots (n-1) + n$. For even $n$, the center of this sum looks like $\ldots + \frac{n}{2}-1 + \frac{n}{2} + \frac{n}{2}+1 + \ldots$. Every element in this sum can be paired with its inverse $\pmod{n}$, except for $\frac{n}{2}$. Thus, the sum equals $\frac{n}{2}$. Since $T_n = \frac{n}{2}$, $\frac{n}{2}! = \frac{n}{2}$. Let $g$ directly precede $\frac{n}{2}$. Then, $g! = \frac{n}{2}! - \frac{n}{2} = \frac{n}{2}-\frac{n}{2} = 0$. Thus, 0 is duplicated, so $\FZ \neq \ZnZ$. 
\end{proof}
\begin{lemma}
For even $n$, $\frac{n}{2}$ cannot be the second smallest element.
\end{lemma}
\begin{proof}
If $\frac{n}{2}$ is the second least element, then $\frac{n}{2}! = 0 + \frac{n}{2} = \frac{n}{2}$. However, we know by Lemma 2 that the factorial of the greatest element of $\ZnZ$ is $\frac{n}{2}$. Thus, if the second least element is $\frac{n}{2}$, then $\frac{n}{2}$ is duplicated, so $\FZ \neq \ZnZ$.
\end{proof}
\end{document}
