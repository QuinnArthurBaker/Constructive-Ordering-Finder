\documentclass{amsart}


\usepackage[usenames]{color}
\usepackage{amsmath, amssymb, latexsym, multicol, color, enumerate,comment,booktabs}
\usepackage{amsthm}
\usepackage{float}
\usepackage{tikz,url}
\usepackage{graphicx, caption}
\usepackage{subcaption}
\captionsetup{compatibility=false}
\usepackage[all]{xy}
\usepackage{amsfonts}

\newcommand{\ag}{\alpha}
\newcommand{\bg}{\beta}
\newtheorem*{claim}{Claim}
\newtheorem{Proposition}{Proposition}
\newtheorem{lemma}{Lemma}
\begin{document}

\title{Integral Bases for Triquadratic Number Fields}
\author{Zack Baker}
\address{Zack Baker,  The King's University, 9125 50 St NW, Edmonton, AB T6B 2H3}
\email{zack.baker@lab.kingsu.ca}

\maketitle

\section{Introduction} 

An \textit{n-quadratic number field}, $n\in\mathbb{N}$, is any field $K$ of degree $2^n$ over $\mathbb{Q}$ that is created by adjoining the square roots of rational, squarefree integers to $\mathbb{Q}$. That is, an $n$-quadratic number field $K$ has the form $K=\mathbb{Q}\left(\sqrt{a_1},\sqrt{a_2},\ldots, \sqrt{a_m}\right)$ for $a_1,...,a_m$ squarefree rational integers.  If $n>1$, the field is also known as a \textit{multiquadratic number field}.

Ideally, we like to define $n$-quadratic number fields by adjoining exactly $n$ square roots to $\mathbb{Q}$. Notice that $\mathbb{Q}\left(\sqrt{2},\sqrt{3}\right)$ and $\mathbb{Q}\left(\sqrt{2},\sqrt{3},\sqrt{6}\right)$ both represent the same 2-quadratic (that is, \textit{biquadratic}) number field, but the first representation is written more concisely, and the presence of $\sqrt{6}$ in this field can be easily obtained by multiplying $\sqrt{2}$ and $\sqrt{3}$. Thus, in this paper, we will always express $n$-quadratic number fields by the adjoining of $n$ square roots to $\mathbb{Q}$. 

This means that whenever we create an $n$-quadratic field $K=\mathbb{Q}\left(\sqrt{a_1},\sqrt{a_2},\ldots, \sqrt{a_n}\right)$, we are assuming $a_1,...,a_n\in\mathbb{Z}$ are squarefree, with the additional property that for any $I\subset\{1,...n\}$, we have that $sf\left(\prod_{i\in I}a_i\right))$ is not in the set $\{a_1,...,a_n\}$ whenever $I$ contains at least two elements. Here, the notation $sf$ refers to the squarefree part of an integer; for example $sf(20)=5$.


The purpose of this paper is to provide a simplified general form for integral bases of all classifications of triquadratic number fields. An \textit{integral basis} for a number field $K$ is a set of $n$ algebraic integers in $K$ such that every element of the ring of integers of $K$, $\mathcal{O}_K$, can be written as an integer linear combination of elements in the integral basis, where $n$ is the degree of $K$. Furthermore, a \textit{normal integral basis} is an integral basis formed of an element and all of its conjugates with respect to $\mathbb{Q}$. The ring of integers of a number field may have several integral bases. We seek to find the simplest integral bases for rings of integers of the different classifications of triquadratic number fields. 

The integral bases for rings of integers of  quadratic fields, $\mathbb{Q}\left(\sqrt{D}\right)$, are well known. If $D \equiv 1 \pmod{4}$, then the integral basis of $\mathcal{O}_{\mathbb{Q}(\sqrt{D})}$ is $\{1, \frac{1+\sqrt{D}}{2}\}$. If $D \not\equiv 1 \pmod{4}$, then the integral basis for $\mathcal{O}_{\mathbb{Q}(\sqrt{D})}$ is $\{1, \sqrt{D}\}$. The integral bases for rings of integers for the different classifications of biquadratic fields are given by Kenneth S. Williams. ~\cite{Wil}. For a biquadratic field $\mathbb{Q}(\sqrt{A},\sqrt{B})$, where $\gcd(A,B) = g$, and $ A_1=A/g$, $B_1= B/G$, the following table describes the integral basis for $\mathcal{O}_{\mathbb{Q}(\sqrt{A},\sqrt{B})}$.\\ \\

\begin{tabular}{|c|c|}
\hline
$\left\{1, \frac{1+\sqrt{a}}{2}, \frac{1+\sqrt{b}}{2}, \frac{1+\sqrt{a}+\sqrt{b}+\sqrt{a_1b_1}}{2}\right\}$ &  $a,b,a_1,b_1\equiv 1 \pmod{4}$\\
\hline
$\left\{1, \frac{1+\sqrt{a}}{2}, \frac{1+\sqrt{b}}{2}, \frac{1-\sqrt{a}+\sqrt{b}+\sqrt{a_1b_1}}{4}\right\}$ & $a,b\equiv 1 \pmod{4}, a_1,b_1\equiv 3 \pmod{4}$\\
\hline $\left\{1, \frac{1+\sqrt{a}}{2}, \sqrt{b}, \frac{\sqrt{b}+\sqrt{a_1b_1}}{2}\right\}$ & $a\equiv 1 \pmod{4}, b\equiv 2 \pmod{4}$\\
\hline
$\{1,\sqrt{a},\sqrt{b},\frac{\sqrt{a} + \sqrt{a_1b_1}}{2}\}$ & $a\equiv 2 \pmod{4}$, $b\equiv 3\pmod{4}$ \\
\hline
$\{1,\sqrt{a}, \frac{\sqrt{a} + \sqrt{b}}{2}, \frac{1+\sqrt{a_1b_1}}{2}\}$ & $a,b\equiv 3 \pmod{4}$\\
\hline
\end{tabular}\\

The results in this paper are summarized and extended from the work of D. Chatelain~\cite{Chat}. This paper aims to make explicit and simplify these results in the case of triquadratic fields, to obtain bases that are easier to understand and apply.


\section{Integral Bases for Triquadratic Fields}

Throughout this section we will take $A_i = \alpha_i^2$, $A_i \in \mathbb{Z}$ squarefree, $1\leq i\leq7$. Further, we will always assume $A_1, A_2, A_3 \equiv 1 \pmod{4}$. When the context is clear, we will use the shorthand $(A_i,A_j)$ in place of $\gcd(A_i,A_j)$, $1\leq i\leq7$.

We define $\alpha_i$, $4\leq i\leq 7$ as follows: 

\[\alpha_4 := \frac{\alpha_1\alpha_2}{(A_1,A_2)},\hspace{0.2in} \alpha_5 := \frac{\alpha_1\alpha_3}{(A_1,A_3)},\]
\[\alpha_6 := \frac{\alpha_2\alpha_3}{(A_2,A_3)},\hspace{0.2in}  \alpha_7 := \frac{\alpha_1\alpha_2\alpha_3}{(A_1,A_2)\cdot(A_3,A_4)}.\]


\subsection{An Integral Basis for $K = \mathbb{Q}\left(\alpha_1,\sqrt{2} \alpha_2,\sqrt{-1}\alpha_3\right)$:}

In this section, we give a simplified version of an integral basis for triquadratic fields that can be written in the form $K = \mathbb{Q}\left(\alpha_1,\sqrt{2} \alpha_2,\sqrt{-1}\alpha_3\right)$. In the original paper by Chatelain, he addresses number fields of the form $L = \mathbb{Q}\left(\alpha_1,\sqrt{-2} \alpha_2,\sqrt{-1}\alpha_3\right)$ separately. However, $L$ can be re-written to be in the same form as $K$, by replacing $\alpha_2$ with $\alpha_6$. We prove this in the following lemma:

 \begin{lemma} Let $L$ be a triquadratic field that is written $L = \mathbb{Q}\left(\nu_1,\sqrt{-2} \nu_2,\sqrt{-1}\nu_3\right)$ with $\nu_i^2\equiv1\pmod4$, $1\leq i\leq 3$. Then there exist $\alpha_1,\alpha_2,\alpha_3\in\mathbb{C}$ such that $L= \mathbb{Q}\left(\alpha_1,\sqrt{2} \alpha_2,\sqrt{-1}\alpha_3\right)$ and $\alpha_i^2\equiv1\pmod4$ are integers, $1\leq i\leq3$.
 \end{lemma}
 
 
 \begin{proof}
Define $N_1 := \nu_1^2$, $N_2 := \nu_2^2$ and $N_3 = \nu_3^2$. Define $\alpha_1:=\nu_1$, $\alpha_3:=\nu_3$ and $\alpha_2:=\frac{-\nu_2\nu_3}{(N_1,N_2)}$. Then, since $\sqrt{-2} \nu_2\cdot\sqrt{-1}\nu_3=\sqrt{2}(-\nu_2\nu_3)$, we clearly have that $L=\mathbb{Q}\left(\alpha_1,\sqrt{2} \alpha_2,\sqrt{-1}\alpha_3\right)$, and we just need to prove that $\alpha_2^2\equiv1\pmod4$ to prove the lemma. 

We have that
\[(\alpha_2)^2=\left(\frac{-\nu_2\nu_3}{(N_1,N_2)}\right)^2=\frac{\nu_2^2\nu_3^2}{(N_2,N_3)^2}=\frac{N_2N_3}{(N_2,N_3)^2}.\]

Clearly the gcd $(N_2,N_3)$ divides both $N_2$ and $N_3$, so $N_2/(N_2,N_3)$ and $N_3/(N_2,N_3)$ are both odd integers. Further, they are congruent to each other modulo 4. Thus their product will be congruent to 1 modulo 4, so $\alpha_2^2\equiv1\pmod4$ as well.
 \end{proof}

\begin{Proposition} Take $\alpha_i$, $1\leq i\leq 7$ to be as defined above. Let $K$ be any triquadratic number field that can be written in the form $\mathbb{Q}\left(\alpha_1,\sqrt{2} \alpha_2,\sqrt{-1}\alpha_3\right)$. Then an integral basis for $\mathcal{O}_K$ is



\begin{align*}
      & \left\{ 1,\sqrt{2}\alpha_2, \sqrt{-1}\alpha_3, \frac{1}{2}\left(1+\alpha_1\right), \frac{\sqrt{2}}{2}\left(\alpha_2 + \alpha_4\right), \frac{\sqrt{2}}{2}\left(\alpha_2 + \sqrt{-1}\alpha_6\right), \right. \\
      &  \left. \frac{\sqrt{-1}}{2}\left(\alpha_3 + \alpha_5 + \sqrt{2}\alpha_6\right), \frac{\sqrt{2}}{4}\left(\alpha_2 + \alpha_4 + \sqrt{-1}\alpha_6 + \sqrt{-1}\alpha_7\right) \right\}
\end{align*}






\end{Proposition}
\begin{proof}
Theorem 10 of ~\cite{Chat} shows us how to find the integral basis for fields of this form. We must first define 8 quantities, $\beta_i$ for $0\leq i \leq 7$:

\begin{center}

\begin{tabular}{cccc}

$\beta_0 = 1$ &  $\beta_1 = \alpha_1$ & $\beta_2 = \sqrt{2}\alpha_2$ &  $\beta_3 = \sqrt{2}\alpha_4$ \\
$\beta_4 = \sqrt{-1}\alpha_3$ & $\beta_5 = \sqrt{-1}\alpha_5$ & $\beta_6  = \sqrt{-2}\alpha_6$ & $\beta_7 = \sqrt{-2}\alpha_7$
\end{tabular}

\end{center}




Using these beta terms, we can begin constructing the integral basis using Chatelain's construction. Four of the basis terms are defined directly by these beta terms, and the other four are given  by their conjugates with respect to the field $\mathbb{Q}\left(\sqrt{2}\alpha_2,\sqrt{-1}\alpha_3\right)$. The four explicit terms are defined as follows:

$$ \gamma_0 :=  \frac12(\beta_0 + \beta_1) = \frac12(1+\alpha_1)$$
$$ \gamma_1 :=  \frac12 (\beta_2 + \beta_3) = \frac12\left(\sqrt{2}\alpha_2 + \sqrt{2}\alpha_4\right)$$
$$ \gamma_2 := \frac12\cdot\sum_{k=4}^{6} \beta_k= \frac12\left(\sqrt{-1}\alpha_3 + \sqrt{-1}\alpha_5+ \sqrt{-2}\alpha_6\right)$$
$$ \gamma_3 := \frac14\left(\sum_{j\in\{2,3,6,7\}} \beta_j \right) = \frac14\left(\sqrt{2}\alpha_2 +\sqrt{2}\alpha_4 +\sqrt{-2}\alpha_6 + \sqrt{-2}\alpha_7\right)$$.


Taking the conjugates of these elements with respect to $\mathbb{Q}\left(\sqrt{2}\alpha_2, \sqrt{-1}\alpha_3\right)$, we get 


$$ \gamma_4 :=  \frac12(1-\alpha_1)$$
$$ \gamma_5 :=  \frac12\left(\sqrt{2}\alpha_2 - \sqrt{2}\alpha_4\right)$$
$$ \gamma_6 :=  \frac12\left(\sqrt{-1}\alpha_3 - \sqrt{-1}\alpha_5 + \sqrt{-2}\alpha_6\right)$$
$$ \gamma_7 :=  \frac14\left(\sqrt{2}\alpha_2 - \sqrt{2}\alpha_4 + \sqrt{-2}\alpha_6 - \sqrt{-2}\alpha_7\right)$$


Then, by ~\cite{Chat}, $\{\gamma_0, \gamma_1,\gamma_2,\gamma_3,\gamma_4,\gamma_5,\gamma_6,\gamma_7\}$ is an integral basis for $\mathcal{O}_k.$


We wish to simplify this basis to a more compact form; and in the remainder of this proof we simplify the above integral basis to be the basis stated in this Proposition. We begin by adding conjugates, and replacing certain basis elements with these sums. To do this define $b_{2i}'=\gamma_i+\gamma_{i+4}$ and $b_{2i+1}'=\gamma_{i+4}$ for $0\leq i\leq 3$.
In particular, 
\[b_0' = \gamma_0 + \gamma_4 =1,\]
\[b_1' = \gamma_0 = \frac{1+\alpha_1}{2},\]
\[b_2' = \gamma_1 + \gamma_5 = \sqrt{2}\alpha_2,\]
\[b_3' = \gamma_1 = \frac{1}{2}\left(\sqrt{2}\alpha_2 + \sqrt{2}\alpha_4\right),\]
\[b_4' = \sqrt{-1} \alpha_3 + \sqrt{-2}\alpha_6,\]
\[b_5' = \gamma_2 = \frac12 \left(\sqrt{-1}\alpha_3 + \sqrt{-1}\alpha_5 + \sqrt{-2}\alpha_6\right),\]
\[b_6'=\gamma_3+\gamma_7 = \frac12\left(\sqrt{2}\alpha_2 + \sqrt{-2}\alpha_6\right),\]
\[b_7' = \gamma_3 =  \frac14\left(\sqrt{2}\alpha_2 + \sqrt{2}\alpha_4 + \sqrt{-2}\alpha_6 + \sqrt{-2}\alpha_7\right).\]


For $i\in\{0,1,2,3,5,6,7\}$, let $b_i:=b_i'$ and let
\begin{align*}
b_4 &:=b_4'-2b_6'+b_2' \\
&= \sqrt{-1} \alpha_3 + \sqrt{-2}\alpha_6 -\sqrt{2}\alpha_2 - \sqrt{-2}\alpha_6 + \sqrt{2}\alpha_2 \\
&= \sqrt{-1} \alpha_3.
\end{align*}
 
Then $\{b_0,b_1,b_2,b_3,b_4,b_5,b_6,b_7\}$ is an integral basis for $\mathcal{O}_K$, thus establishing the proposition.
  
\end{proof}
 
 
 
 \textbf{Example:} Let $K = \mathbb{Q}\left(\sqrt{-15},\sqrt{-6},\sqrt{7}\right) = \mathbb{Q}\left(\sqrt{-15},\sqrt{2}\sqrt{-3},\sqrt{-1}\sqrt{-7}\right)$. Then, $\alpha_1 = \sqrt{-15}$, $\alpha_2 = \sqrt{-3}$, and $\alpha_3 = \sqrt{-7}$. Using the definitions above, an integral basis for $\mathcal{O}_K$ is
 \begin{align*}
      & \left\{ 1, \sqrt{-6}, \sqrt{7},\frac{1}{2}\left(1+\sqrt{15}\right),  \frac{1}{2}\left(\sqrt{-6}+\sqrt{10}\right), \frac{1}{2} \left(\sqrt{-6} + \sqrt{-42} \right),\right. \\
      & \left. \frac{1}{2}\left(\sqrt{7} + \sqrt{42} + \sqrt{-105}\right), \frac{1}{4}\left(\sqrt{-6} + \sqrt{10} + \sqrt{-42} + \sqrt{70}\right)\right\}
\end{align*}
 
 
\newpage\subsection{An Integral Basis for $K=\mathbb{Q}\left(\alpha_1,\alpha_2,\alpha_3\right)$}

\begin{Proposition} Take $\alpha_i$, $1\leq i\leq 7$ to be as defined above. Let $K$ be any triquadratic number field that can be written in the form $\mathbb{Q}\left(\alpha_1, \alpha_2,\alpha_3\right)$. Then an integral basis for $\mathcal{O}_K$ is


\begin{align*}
      & \left\{1, \frac{1}{2}(1+\alpha_1), \frac{1}{2}(1+\alpha_2), \frac{1}{2}(1+\alpha_3), \frac{1}{4}(1+\alpha_2 + \alpha_3 + \alpha_6), \frac{1}{4}(1+\alpha_1 + \alpha_3 + \alpha_5), \right. \\
      & \left.\frac{1}{4}(1 + \alpha_1 + \alpha_2 + \alpha_4), \frac{1}{8}(1 + \alpha_1 + \alpha_2 + \alpha_3 + \alpha_4 + \alpha_5 + \alpha_6 + \alpha_7)\right\}
\end{align*}





\end{Proposition}
\begin{proof} Theorem 9b of ~\cite{Chat} gives us the integral basis of the field.  A normal integral basis is guaranteed for these fields. The basis elements of the normal basis are given by the conjugates of $\gamma_0 = \frac{1}{8}+\sum_{i=1}^{7} \frac{\alpha_i}{8} = \frac{1}{8}\left(1+\alpha_1+\alpha_2+\alpha_3+\alpha_4+\alpha_5+\alpha_6+\alpha_7\right)$ with respect to $\mathbb{Q}$. Its conjugates are:
$$ \gamma_1 = \frac{1}{8}\left(1-\alpha_1+\alpha_2+\alpha_3-\alpha_4 - \alpha_5 + \alpha_6 - \alpha_7\right)$$
$$ \gamma_2 = \frac{1}{8}\left(1+\alpha_1-\alpha_2+\alpha_3-\alpha_4 + \alpha_5 - \alpha_6 - \alpha_7)\right)$$
$$ \gamma_3 = \frac{1}{8}\left(1+\alpha_1+\alpha_2-\alpha_3+\alpha_4 - \alpha_5 - \alpha_6 - \alpha_7\right)$$
$$\gamma_4 = \frac{1}{8}\left(1-\alpha_1-\alpha_2+\alpha_3+\alpha_4 - \alpha_5 - \alpha_6 + \alpha_7\right)$$
$$\gamma_5 = \frac{1}{8}\left(1-\alpha_1+\alpha_2-\alpha_3-\alpha_4 + \alpha_5 - \alpha_6 + \alpha_7\right)$$
$$\gamma_6 =  \frac{1}{8}\left(1+\alpha_1-\alpha_2-\alpha_3-\alpha_4 - \alpha_5 + \alpha_6 + \alpha_7\right)$$
$$\gamma_7 =  \frac{1}{8}\left(1-\alpha_1-\alpha_2-\alpha_3+\alpha_4 + \alpha_5 + \alpha_6 - \alpha_7 \right)$$
Then, $\{\gamma_0,\gamma_1,\gamma_2,\gamma_3,\gamma_4,\gamma_5,\gamma_6,\gamma_7\}$ forms a normal integral basis for $\mathcal{O}_K$. We can further simplify this basis. Define $b_i$, $0\leq i\leq 7$ as follows:
$$b_0 = \sum_{i=0}^7 \gamma_i = 1$$
$$b_1 = \gamma_0+\gamma_2+\gamma_3+\gamma_6 = \frac{1}{2}(1+\alpha_1)$$
$$b_2 = \gamma_0 + \gamma_1 +\gamma_3 + \gamma_5 = \frac{1}{2}(1+\alpha_2)$$
$$b_3 = \gamma_0 + \gamma_1 + \gamma_2 + \gamma_4 = \frac{1}{2}(1+\alpha_3)$$
$$b_4 = \gamma_0 + \gamma_1 = \frac{1}{4}(1+\alpha_2 + \alpha_3 + \alpha_6)$$
$$b_5 = \gamma_0 + \gamma_2  = \frac{1}{4}(1+\alpha_1 + \alpha_3 + \alpha_5)$$
$$b_6 = \gamma_0 + \gamma_3 =  \frac{1}{4}(1+\alpha_1 + \alpha_2 + \alpha_4)$$
$$b_7 = \gamma_0  = \frac{1}{8}(1+\alpha_1 + \alpha_2 + \alpha_3 + \alpha_4 + \alpha_5 + \alpha_6 + \alpha_7)$$
Then, $\{b_0,b_1,b_2,b_3,b_4,b_5,b_6,b_7\}$ forms an integral basis for $\mathcal{O}_K$. 
\end{proof}
\textbf{Example:} Let $K=\mathbb{Q}\left(\sqrt{5},\sqrt{13},\sqrt{-3}\right).$ Then, $\alpha_1 = \sqrt{5}$, $ \alpha_2 = \sqrt{13}$, and $\alpha_3 = \sqrt{-3}$. Using the definitions above, an integral basis for $\mathcal{O}_K$ is 
\begin{align*}
      & \left\{1,\frac{1}{2}(1+\sqrt{-3}), \frac{1}{2}(1+\sqrt{5}),\frac{1}{2}(1+\sqrt{13}),  \frac{1}{4}(1 + \sqrt{-3} + \sqrt{13} + \sqrt{-39}),\frac{1}{4}(1+ \sqrt{-3} + \sqrt{5} + \sqrt{-15}), \right. \\
      & \left.\frac{1}{4}(1 + \sqrt{5} + \sqrt{13} + \sqrt{65}), \frac{1}{8}(1+\sqrt{-3} + \sqrt{5} + \sqrt{13} + \sqrt{-15} + \sqrt{-39} + \sqrt{65} + \sqrt{-195}) \right\}
\end{align*}

\newpage
\subsection{An Integral Basis for $K=\mathbb{Q}\left(\alpha_1,\alpha_2,\delta \alpha_3\right), \delta \in \{\sqrt{2},\sqrt{-1}\}$}

\begin{Proposition}
Take $\alpha_i$, $1\leq 7$ to be as defined above. Let $K$ be any triquadratic number field that can be written in the form $\mathbb{Q}\left(\alpha_1,\alpha_2,\delta\alpha_3\right)$, for $\delta \in \{\sqrt{2}\sqrt{-1}\}$. Then an integral basis for $\mathcal{O}_K$ is 


\begin{align*}
      & \left\{1,\frac{1}{4}(1+\alpha_1),\frac{1}{4}(1+\alpha_2),\frac{1}{4}(1+\alpha_4), \frac{1}{2}(\alpha_3 + \delta\alpha_5),  \right. \\
      &  \left. \frac{1}{2}(\alpha_3 + \delta\alpha_6),\frac{1}{4}(\alpha_4 + \delta\alpha_5 + \delta\alpha_6 + \delta\alpha_7), \frac{1}{8}(1+\alpha_1+\alpha_2+\alpha_4) \right\}
\end{align*}
\end{Proposition}
\begin{proof}Theorem 11  of ~\cite{Chat} shows us how to find an integral basis for $\mathcal{O}_K$. First, we find beta terms:
$$\beta_0 = 1$$
$$\beta_1 = \alpha_1$$
$$\beta_2 = \alpha_2$$
$$\beta_3 = \alpha_4$$
$$\beta_4 = \delta \alpha_3$$
$$\beta_5 = \delta\alpha_5$$
$$\beta_6 = \delta\alpha_6$$
$$\beta_7 = \delta\alpha_7$$
To find the integral basis, we calculate $\gamma_0$ and $\gamma_1$:
$$\gamma_0 = \sum_{i=0}^3 \frac{\beta_i}{8} = \frac{1}{8}\left(1+\alpha_1 + \alpha_2 + \alpha_4 \right),$$
$$\gamma_1 = \sum_{j=4}^7 \frac{\beta_i}{4} = \frac{1}{4}\left(\alpha_3 + \delta \alpha_5 + \delta\alpha_6 + \delta\alpha_7\right).$$
These two elements with their conjugates with respect to $\mathbb{Q}\left(\delta\alpha_3\right)$ form the integral basis of the ring of integers of fields of this form. The conjugates of these elements with respect to $\mathbb{Q}(\delta \alpha_3)$ are:
$$ \gamma_2 := \frac{1}{8} \left ( 1-\alpha_1 + \alpha_2 - \alpha_4\right)$$
$$ \gamma_3 := \frac{1}{8} \left (1 + \alpha_1 - \alpha_2 - \alpha_4\right)$$
$$ \gamma_4 := \frac{1}{8} \left (1 - \alpha_1 - \alpha_2 + \alpha_4\right)$$
$$ \gamma_5 := \frac{1}{4}\left(\alpha_3 - \delta\alpha_5 + \delta\alpha_6 - \delta\alpha_7\right)$$
$$ \gamma_6 := \frac{1}{4}\left(\alpha_3 + \delta\alpha_5 - \delta\alpha_6 - \delta\alpha_7\right)$$
$$ \gamma_7 := \frac{1}{4}\left(\alpha_3 - \delta\alpha_5 - \delta\alpha_6 + \delta\alpha_7\right).$$
Then, $\{\gamma_0, \gamma_1, \gamma_2, \gamma_3,\gamma_4, \gamma_5, \gamma_6, \gamma_7\}$ forms an integral basis for $\mathcal{O}_K$. We can further simplify this basis. For $b_i$, $0 \leq i \leq 7$, let
$$ b_0 := \gamma_1 + \gamma_5 + \gamma_6 + \gamma_7 = 1$$
$$b_1 := \gamma_1 + \gamma_5 = \frac{1}{2}\left(\alpha_3 + \delta\alpha_6\right)$$
$$b_2 := \gamma_1 + \gamma_6 = \frac{1}{2}\left(\alpha_3 + \delta \alpha_5\right)$$
$$b_3 := \gamma_0 + \gamma_3 = \frac{1}{4}\left(1 + \alpha_1\right)$$
$$b_4 := \gamma_0 + \gamma_2 = \frac{1}{4}\left(1 + \alpha_2\right)$$
$$b_5 := \gamma_0 + \gamma_4 = \frac{1}{4}\left(1 + \alpha_4\right)$$
$$b_6 := \gamma_0 = \frac{1}{8}\left(1 + \alpha_1 + \alpha_2 + \alpha_4\right)$$
$$b_7 := \gamma_1 = \frac{1}{4}\left(\alpha_3 + \delta\alpha_5 + \delta\alpha_6 + \delta \alpha_7\right)$$
Then, $\{b_0,b_1,b_2,b_3,b_4,b_5,b_6,b_7\}$ forms an integral basis for $\mathcal{O}_K$. \end{proof}

\textbf{Example 1:} Let $K = \mathbb{Q}\left(\sqrt{-3},\sqrt{-7},\sqrt{26}\right) = \mathbb{Q}\left(\sqrt{-3},\sqrt{-7},\sqrt{2}\sqrt{13}\right)$. Then, $\alpha_1 = \sqrt{-3}$, $\alpha_2 = \sqrt{-7}$, $\alpha_3 = \sqrt{13}$, and $\delta = \sqrt{2}$. Using the definitions above, an integral basis for $\mathcal{O}_K$ is 
\begin{align*}
      & \left\{1, \frac{1}{2}(\sqrt{13} + \sqrt{-78}), \frac{1}{2}(\sqrt{13} + \sqrt{-182}),  \frac{1}{4}(1 + \sqrt{-3}), \frac{1}{4}(1 + \sqrt{-7}), \right. \\
      &  \left. \frac{1}{4}(1 + \sqrt{21}), \frac{1}{4}(\sqrt{13} + \sqrt{-78} + \sqrt{-182} + \sqrt{546}), \frac{1}{8}(1 + \sqrt{-3} + \sqrt{-7} + \sqrt{21}) \right\}
\end{align*}
\textbf{Example 2:} Let $K = \mathbb{Q}\left(\sqrt{-7},\sqrt{-15}, \sqrt{-13}\right) = \mathbb{Q}\left(\sqrt{-7},\sqrt{-15},\sqrt{-1}\sqrt{13}\right)$. Then, $\alpha_1 = \sqrt{-7},$ $\alpha_2 = \sqrt{-15}$, $\alpha_3 = \sqrt{13}$, and $\delta = \sqrt{-1}$. By the definitions above, an integral basis for $\mathcal{O}_K$ is 

\begin{align*}
      & \left\{1, \frac{1}{2}\left(\sqrt{13} + \sqrt{91}\right), \frac{1}{2}\left(\sqrt{13} + \sqrt{195}\right), \frac{1}{4}\left(1+\sqrt{-7}\right), \frac{1}{4}\left(1+\sqrt{-15}\right), \right. \\
      &  \left. \frac{1}{4}\left(1+\sqrt{105}\right), \frac{1}{4}\left(\sqrt{13} + \sqrt{91} + \sqrt{195} + \sqrt{-1365}\right), \frac{1}{8}\left(1 + \sqrt{-7} + \sqrt{-15} + \sqrt{105}\right) \right\}
\end{align*}

\begin{thebibliography}{}

 \bibitem{Chat} Chatelain, D., \textit{Bases des Entiers des Corps Compos\'es Par Des Extensions Quadratiques de $\mathbb{Q}$}, Ann. Univ. Besan\c{c}on, Math. \textbf{6} (1973)
 \bibitem{Wil} Williams, Kenneth S. \textit{Integers of biquadratic fields.} Canadian Mathematical Bulletin Bulletin canadien de mathématiques 13.0 (1970): 519-26. Web.
 \end{thebibliography}
 


\end{document}
