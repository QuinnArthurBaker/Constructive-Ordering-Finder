\documentclass{article}
\usepackage{amsmath}
\usepackage{amsfonts}
\usepackage{amsthm}
\title{Bases whole body composed by quadratic extensions of $\mathbb{Q}$}
\begin{document}
\maketitle
{\Large Introduction} \\
The bodies studied are extensions of $k_n$of the field of rationals composed of n quadratic extensions of $\mathbb{Q}$, defined by their discriminant $(D_\lambda)_{\lambda \in \Lambda}$
\par It is proposed to explicitly determine the bases on $\mathbb{Z}$ of the ring of integers of $k_n$. 
\par For this $k_n$ is associated with the extension field $K$ obtained by adjoining to $\mathbb{Q}$ square roots of the first rational integers and congruent to 1 modulo 4, dividing the discriminants $D_\lambda$ and the $\sqrt{2}$ numbers or $i = \sqrt{-1}$ if are discriminating congruent 2 or -1 modulo 4.
\par We know that the product bases on $\mathbb{Z}$ rings of integers of two bodies of numbers with discriminant are relatively prime is a base $\mathbb{Z}$ of the ring of integers of the body consists. This result allows to determine the rings of integers of the body consists. This result allows to determine the $K$ ring of integers and the ring of integers of the subfield $k$ of $K$.
\par Include in particular the condition of existence of a normal basis of whole (comprised of conjugated relative to $\mathbb{Q}$ the same integer). We show that this base is unique when it exists (the sign near one of its generators) and given expression. It is whown that in all cases the discriminant of $k_n$ of $K$ is equal to the product of discriminating all in quadratic fields $k_n$. These results, generalize those obtained by Kenneth S. Williams in the case n=2 ([1]). \\ \\
{\Large Chapter 1: Construction and properties of the field $k_n$ } \\
{\large Paragraph 1: How to define the field $k_n$}
\par $n$ quadratic fields which generate $k_n$ is assumed given.
\par We take for generator of each one of the two numbers whose square is rational integer without square factors (squarefree).
\par Note $\alpha_{2^k} (0\leq k \leq n-1)$ generators and $A_{2^k} (0\leq k \leq n-1)$ their squares
\par (The indicies $2^k$ will be justified by the introduction of numbers $\alpha_i$ and $A_i$ definied for indices $i$ not power of 2)
\par No one expresses numbers $\alpha_{2^k}$ is superfluous by the conditions: 
$$ \alpha_{2^k} \not\in \mathbb{Q}(\alpha_{2^0}, \alpha_{2^1}, \ldots \alpha_{2^{k-1}})\ \ (1 \leq k \leq n-1)$$ 
\par or the equivalent relationships
$$ \mathbb{Q}(\alpha_{2^k}) \cap \mathbb{Q}(\alpha_{2^0}, \ldots, \alpha_{2^{k-1}}) = \mathbb{Q}\ \ (1\leq k \leq n-1)$$
\par It follows the conditions (1') that the field $k_n = \mathbb{Q}(\alpha_{2^0}, \ldots, \alpha_2^{n-1})$ is a Galoi extension of $\mathbb{Q}$ of degree $2^n$ whose Galois group $G_n$ id isomorphic to the Cartesian product of n Galois groups of $\mathbb{Q}$ of $\mathbb{Q}(\alpha_{2^k})$(for $ 0 \leq k \leq n-1)$([2]). $G_n$ is the direct product of $n$ subgroups of order 2. All its components except the neutral element Id. are of order 2 and is Abelian. 
\par Conversely, if we define $k_n$ as an abelian extension of $\mathbb{Q}$ of degree $2^n$ whose Galois group is equal to the direct product $g_0 \times g_1 \times \ldots \times g_{n-1}$ $n$ subgroups of order 2, $k_n$ is spanned by the $n$ generators quadratic extensions $\mathbb{Q}$ fixed body of $g_0 \times \ldots \times {id} \times \ldots \times g_{n-1}$ or {id} subgroups replace $g_k$ for $0 \leq k \leq n-1$ ([2]).\\ 
{\large Paragraph 2: Choosing a base $k$ on $\mathbb{Q}$}
\par J for non power of 2 and j we define little by little the numbers $A_j$ from the numbers $A_{2^k}\ \ (0\leq k \leq n-1)$ posing for $j = i +2^k, \ \ 0<i<2^k, \ \ 1\leq k \leq n-1$:
$$ A_j = \dfrac{A_i \cdot A_{2^k}}{(d_{i,2^k})^2}, $$
\par or is one of two GCD of $A_i$ and $A_{2^k}$. It will be specified in Chapter 2 the choice of the sign of $d_{i,2^k}$. It then defines the numbers has relations for j = $i + 2^k$, $0<i<2^k$, $1\leq k \leq n-1$:
$$ \alpha_j = \dfrac{\alpha_i \cdot \alpha_{2^k}}{d_{i,2^k}}\ \ .$$
\par Furthermore, it asks: $\alpha_0 = A_0 = 1$
\par $A_j$ ($0\leq j \leq 2^n-1$) numbers are rational integers without edges(square?) factors and is one of the roots of the polynomial $X^2 - A_j$. 
$$\text{Lemma 1:}\ \ N = \{\alpha_j ; 0\leq j \leq 2^n-1\} \text{ is a base for } k \text{ on } \mathbb{Q}$$
\par $k_n$ is indeed the body consists of $\mathbb{Q}(\alpha_{2^k}$ body shall check ballast conditions (1). It therefore has a base product bases $\{1,\alpha{2^k}\}$, ($0\leq k \leq n-1$). The elements of this base are of the form $\lambda_j \alpha_j$, ($0\leq j \leq 2^n-1$) with $\lambda_j \in \mathbb{Z}-\{0\}$; $\{\alpha_j ; 0 \leq j \leq 2^n-1 \}$ together is also a $k\ \mathbb{Q}$ basic. \\
{\large Paragraph 3: Galois group $G$ of $k_n$ on $Q$}
\par $\mathbb{Q}$-automorphism $\sigma$ of $k_n$ is defined by the given $\sigma(\alpha_{2^k})$ for $0\leq k \leq n-1$. (The values of $\sigma$ for other elements of the basis of $N$ of $k_n$ on $\mathbb{Q}$ are deduced by relations (3)).
\par The numbers ($\alpha_{2^k}^2$) belonging to $\mathbb{Z}$, are invariant under $\sigma$; therefore we have for $\sigma \in G_n$ and $0\leq k \leq n-1$:
$$ \sigma(\alpha_{2^k}) = \pm \alpha_{2^k}$$
\par Let $\sigma_0$ the same application $k_n$.
\par $\sigma_{2^p}$ note the elements of $G_n$ defined by $0\leq p \leq n-1$ on 
$$ \sigma_{2^p} (\alpha 2^p) = -\alpha_{2^p}$$
$$ \sigma_{2^p}(\alpha_{2^p}) = \alpha_{2^k} \text{ for } k\neq p \text{ and } 0 \leq k \leq n-1$$
\par Let $g_p$ subgroup $\{\sigma_0,\sigma_{2^p}\}$ of $G_n$.
\par $G_n$ is equal to the direct product of $g_0 \times g_1 \times \ldots \times g_{n-1}$ its subgroups $g_p $($0 \leq p \leq n-1$).
\par This results from the isomorphism between $G_n$ and the Cartesian product $h_0 \times h_1 \times \ldots h_{n-1}$ Galois groups $\mathbb{Q}(\alpha_{2^p})$, this isomorphism maps has an element of $G_n$, the $n$-tuple of its restrictions on the field $\mathbb{Q}(\alpha_{2^p})$, and the subgroup $g_p$ the subgroup:
$$ \{id\} \times \ldots \times h_p \times \{id\} \times \ldots \times \{id\}$$
\par Then ask for $j=i+2^k$, $0 < i < 2^k$ and $1 \leq k \leq n-1$
$$\sigma_j = \sigma_i \circ \sigma_{2^k}$$
\par $G_n$ is equal to all $sigma_i$ for $0 \leq i \leq 2^n-1$ and we have:
$$ \sigma_i (\alpha_j) = \pm \alpha_j (car(\alpha_j)^2 \in \mathbb{Z})$$
\end{document}